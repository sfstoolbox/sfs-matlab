% Latex macros
%
% This file contains macros for
%   * References in a paper
%   * Defining vectors, signals, etc.
%   * Math
%
% AUTHOR: Sascha Spors, Hagen Wierstorf

%=============== Important command (used also below) =========================
% Overwrite the \vec command to use bold font instead of arrow above the symbol
\renewcommand{\vec}[1]{\ensuremath{ \mathbf{#1} }}


%=============== some macros =================================================
\newcommand{\eg}[0]{e.\,g.}
\newcommand{\qc}[0]{\;,}
\newcommand{\qp}[0]{\;.}
\newcommand{\threeD}[0]{\text{3D}}
\newcommand{\twoD}[0]{\text{2D}}
\newcommand{\twohalfD}[0]{\text{2.5D}}


%=============== References ==================================================
\newcommand{\FigRef}[1]{Fig.~\ref{#1}}
\newcommand{\TblRef}[1]{Table~\ref{#1}}
\newcommand{\SecRef}[1]{Section~\ref{#1}}
\newcommand{\AppRef}[1]{Appendix~\ref{#1}}
\newcommand{\PRef}[1]{Page~\pageref{#1}}
\newcommand{\EqRef}[1]{(\ref{#1})}

% mark some text (need package ``color'')
%\newcommand{\RED}[1]{\textcolor{red}{#1}}
%\newcommand{\GREEN}[1]{\textcolor{green}{#1}}
%\newcommand{\BLUE}[1]{\textcolor{blue}{#1}}

\newcommand{\ul}[1]{\underline{#1}}



%=============== Signale =========================================
\newcommand{\fts}[1]{\ensuremath{ \tilde{#1} }}           % fourier transform expansion coeff
\newcommand{\fss}[1]{\ensuremath{ \mathring{#1} }}        % fourier series expansion coeff
\newcommand{\pws}[1]{\ensuremath{ \bar{#1} }}             % plane wave expansion coeff
\newcommand{\wds}[1]{\ensuremath{ \tilde{#1} }}           % wave domain expansion coeff
\newcommand{\cys}[1]{\ensuremath{ \breve{#1} }}           % cylindric expansion coeff


%=============== Operatoren =========================================
\DeclareMathOperator{\PWT}{\mathcal{P}}
\DeclareMathOperator{\CHT}{\mathcal{C}}
\newcommand{\FT}{\ensuremath{\mathcal{F}}}
\newcommand{\IFT}{\ensuremath{\mathcal{F}^{-1}}}
\DeclareMathOperator{\HT}{\mathcal{H}}
\DeclareMathOperator{\Sys}{\mathcal{S}}


\newcommand{\FS}{\ensuremath{\text{FS}}}
\newcommand{\IFS}{\ensuremath{\text{FS}^{-1}}}
\DeclareMathOperator{\DHT}{\text{DFBT}}
\DeclareMathOperator{\DFT}{\text{DFT}}
\DeclareMathOperator{\IDFT}{\text{IDFT}}


\DeclareMathOperator{\conv}{\ast}
\DeclareMathOperator{\pconv}{\circledast}

\newcommand{\real}[1]{\ensuremath{ \Re\{#1\} }}
\newcommand{\imag}[1]{\ensuremath{ \Im\{#1\} }}


%=============== Koordinatensysteme =========================================
\newcommand{\cc}{\ensuremath{\mathtt{C}}}
\newcommand{\spc}{\ensuremath{\mathtt{H}}}
\newcommand{\yc}{\ensuremath{\mathtt{Y}}}
\newcommand{\pc}{\ensuremath{\mathtt{P}}}

\newcommand{\inc}{{\ensuremath{\text{(1)}}}}
\newcommand{\outg}{{\ensuremath{\text{(2)}}}}

\newcommand{\si}[1]{{\ensuremath{\text{(#1)}}}}



%=============== Funktionen =========================================
\DeclareMathOperator{\circf}{\text{circ}}
\DeclareMathOperator{\sinc}{\text{sinc}}

\newcommand{\sha}{\bot \!\! \bot \!\! \bot}

%=============== Variables =========================================
% omega
% -----
%   c
\newcommand{\omegac}{\ensuremath{ \frac{\omega}{c} }}
% x-x_0 (vector)
\newcommand{\xx}{\ensuremath{ \vec{x}-\vec{x}_0 }}
% x (vector) 
\newcommand{\x}{\ensuremath{ \mathbf{x} }}
% k (vector) 
% NOTE: the \k command is for the ogonek accent, which we very probably will 
% not need, therefore I overwrite it
\renewcommand{\k}{\ensuremath{ \mathbf{k} }}

%=============== Bessel functions ========================================
% Hankel functions H^kind_order
\newcommand{\hankelzero}[1]{\ensuremath{ H_{#1}^{(0)} }}
\newcommand{\hankelone}[1]{\ensuremath{ H_{#1}^{(1)} }}
\newcommand{\hankeltwo}[1]{\ensuremath{ H_{#1}^{(2)} }}


%\DeclareSymbolFont{cyrletters}{OT2}{wncyr}{m}{n}
%\DeclareMathSymbol{\sha}{\mathalpha}{cyrletters}{"58}

%=============== Matrizen & Vektoren =========================================
\newcommand{\Mat}[1]{\ensuremath{ \mathbf{#1} }}
\newcommand{\FSc}[1]{\ensuremath{ \underline{#1} }}
\newcommand{\FVek}[1]{\ensuremath{ \underline{\mathbf{#1}} }}
\newcommand{\FMat}[1]{\ensuremath{ \underline{\mathbf{#1}} }}
\newcommand{\FFSc}[1]{\ensuremath{\underline{\underline{#1}}}}
\newcommand{\FFVek}[1]{\ensuremath{\underline{\underline{\mathbf{#1}}}}}
\newcommand{\FFMat}[1]{\ensuremath{\underline{\underline{\mathbf{#1}}}}}
%\newcommand{\FFSc}[1]{\ensuremath{\uuline{#1}}}
%\newcommand{\FFVek}[1]{\ensuremath{\uuline{\mathbf{#1}}}}
%\newcommand{\FFMat}[1]{\ensuremath{\uuline{\mathbf{#1}}}}
\newcommand{\TP}[1]{\ensuremath{ #1^T }}
\newcommand{\HE}[1]{\ensuremath{ #1^H }}
\newcommand{\norm}[1]{\ensuremath{ \left\| #1 \right\|}}
\newcommand{\ABS}[1]{\ensuremath{ \left| #1 \right|}}
\newcommand{\dotprod}[2]{{\ensuremath{\langle#1,#2\rangle}}}
\newcommand{\diag}[1]{\ensuremath{\text{diag} \{ #1 \} }}
\newcommand{\Bdiag}[1]{\ensuremath{\text{Bdiag} \{ #1 \} }}
\newcommand{\mvec}[1]{\ensuremath{\text{vec} \{ #1 \} }}
\newcommand{\rank}[1]{\ensuremath{\text{rk} \{ #1 \} }}
\newcommand{\tr}[1]{\ensuremath{\text{tr} \{ #1 \} }}

% ----------- Mathematische Operationen ------------
\newcommand{\p}[1]{\ensuremath{\frac{\partial}{\partial #1}}}

% ----- Compatibility to old papers --------------------------------------
\newcommand{\Vek}[1]{\ensuremath{ \mathbf{#1} }}
